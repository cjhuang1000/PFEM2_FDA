\bibitem{cpi1} FDA’s ``Critical Path'' Computational Fluid Dynamics (CFD)/Blood Damage Project: Computational Round Robin problems
https://nciphub.org/wiki/FDA\_CFD
\bibitem{cpi} FDA Critical Path Initiative (CPI) \\\small{https://www.fda.gov/scienceresearch/specialtopics/criticalpathinitiative/default.htm}
\bibitem{fda_res} Hariharan, P., Giarra, M., Reddy, V., Day, S.W.,
Manning, K.B., Deutsch, S., Stewart, S.F., Myers, M.R., Berman, M.R., Burgreen, G.W. and Paterson, E.G. (2011).
Multilaboratory particle image velocimetry analysis of the FDA benchmark nozzle model to support validation of computational
fluid dynamics simulations. Journal of Biomechanical Engineering, 133(4), 041002.
\bibitem{fda_nozzle} Herbertson, L. H., Olia, S. E., Daly, A., Noatch, C. P., Smith, W. A., Kameneva, M. V., \& Malinauskas, R. A. (2015).
Multilaboratory Study of Flow‐Induced Hemolysis Using the FDA Benchmark Nozzle Model. Artificial organs, 39(3), 237-248.
\bibitem{fda_pump} Giarra, M. N. (2009). Shear Stress Distribution and Hemolysis Measurements in a Centrifugal Blood Pump. Rochester Institute of
Technology.
\bibitem{fda_numrob} Zmijanovic, V., Mendez, S., Moureau, V., \& Nicoud, F. (2017). About the numerical robustness of biomedical benchmark cases:
Interlaboratory FDA's idealized medical device. International journal for numerical methods in biomedical engineering, 33(1).
\bibitem{hariharan_nozzle} Hariharan, P., D’Souza, G. A., Horner, M., Morrison, T. M., Malinauskas, R. A., \& Myers, M. R. (2017). Use of the FDA nozzle
model to illustrate validation techniques in computational fluid dynamics (CFD) simulations. PloS one, 12(6), e0178749.
\bibitem{nassau_pump} Nassau, C. J., Wray, T. J., \& Agarwal, R. K. (2015, July). Computational Fluid Dynamic Analysis of a Blood Pump: An FDA
Critical Path Initiative. In ASME/JSME/KSME 2015 Joint Fluids Engineering Conference (pp. V002T26A002-V002T26A002).
American Society of Mechanical Engineers.
\bibitem{heck_hemo} Heck, M. L., Yen, A., Snyder, T. A., O'Rear, E. A., \& Papavassiliou, D. V. (2017). Flow‐Field Simulations and Hemolysis
Estimates for the Food and Drug Administration Critical Path Initiative Centrifugal Blood Pump. Artificial organs, 41(10).
\bibitem{stewart_cfd} Stewart, S. F., Paterson, E. G., Burgreen, G. W., Hariharan, P., Giarra, M., Reddy, V., Stewart, S.F., Paterson, E.G., Burgreen,
G.W., Hariharan, P., Giarra, M., Reddy, V., Day, S.W., Manning, K.B., Deutsch, S., Berman, M.R. and Myers, M.R. (2012).
Assessment of CFD performance in simulations of an idealized medical device: results of FDA’s first computational
interlaboratory study. Cardiovascular Engineering and Technology, 3(2), 139-160.
\bibitem{mali_cfd} Malinauskas, R. A., Hariharan, P., Day, S. W., Herbertson, L. H., Buesen, M., Steinseifer, U., Aycock, K.I., Good, B.C., Deutsch,
S., Manning, K.B. and Craven, B.A. (2017). FDA benchmark medical device flow models for CFD validation. ASAIO Journal,
63(2), 150-160.
\bibitem{gallagher_exp} Gallagher, M. B., Aycock, K. I., Craven, B. A., \& Manning, K. B. (2018). Steady flow in a patient-averaged inferior vena cava—part I: particle image velocimetry measurements at rest and exercise conditions. Cardiovascular engineering and technology, 9(4), 641-653.
% Gallagher, M.B., Aycock, K.I., Craven, B.A. et al. Cardiovasc Eng Tech (2018) 9: 641. https://doi.org/10.1007/s13239-018-00390-2
\bibitem{craven_cfd} Craven, B. A., Aycock, K. I.,\& Manning, K. B. (2018). Steady Flow in a Patient-Averaged Inferior Vena Cava—Part II: Computational Fluid Dynamics Verification and Validation. Cardiovascular engineering and technology, 9(4), 654-673.
%Craven, B.A., Aycock, K.I. \& Manning, K.B. Cardiovasc Eng Tech (2018) 9: 654. https://doi.org/10.1007/s13239-018-00392-0
\bibitem{sph} Monaghan JJ (1988) An introduction to SPH. Comput Phys Com-
mun 48:89–96
\bibitem{pic} Harlow FH (1955) A machine calculation method for hydro-
dynamic problems. Los Alamos Scientific Laboratory Report
LAMS-1956
\bibitem{mac} Harlow FH, Welch J (1965) Numerical calculation of time depen-
dent viscous incompressible flow of fluid with free surface. Phys
Fluids 8(12):2182–2189
\bibitem{mpm} Wieckowsky Z (2004) The material point method in large
strain engineering problems. Comput Methods Appl Mech Eng
193(39):4417–4438
\bibitem{sergio:pfem} Idelsohn SR, Oñate E, Del Pin F (2004) The particle finite element
method a powerful tool to solve incompressible flows with free-
surfaces and breaking waves. Int J Numer Methods 61:964–989

\bibitem{sergio:xivs1} Idelsohn SR, Nigro NM, Limache A, Oñate E (2012) Large
time-step explicit integration method for solving problems with
dominant convection. Comput Methods Appl Mech Eng 217–
220:168–185

\bibitem{sergio:xivs2} Idelsohn SR, Nigro NM, Gimenez JM, Rossi R, Marti J (2013)
A fast and accurate method to solve the incompressible Navier–
Stokes equations. Eng Comput 30(2):197–222

\bibitem{gimenez:parallel} Gimenez JM, Nigro NM, Idelsohn SR (2014) Evaluating the perfor-
mance of the particle finite element method in parallel architectures.
J Comput Part Mech 1(1):103–116

\bibitem{sergio:pfem2_lts} Idelsohn SR, Marti J, Becker P, Oñate E (2014) Analysis of mul-
tifluid flows with large time steps using the particle finite element
method. Int J Numer Methods Fluids 75(9):621–644

\bibitem{gimenez:fs} Gimenez JM, Gonzlez LM (2015) An extended validation of the
last generation of particle finite element method for free surface
flows. J Comput Phys 284:186–205
\bibitem{pablo:FSI} Becker P, Idelsohn SR, Oñate E (2014) A unified monolithic
approach for multi-fluid flows and fluid-structure interaction using
the particle finite element method with fixed mesh. Comput Mech
55(6):1091–1104
\bibitem{gimenez:st} Gimenez JM, Nigro N, Oñate E, Idelsohn S (2016) Surface tension
problems solved with the particle finite element method using large
time-steps. Comput Fluids
\bibitem{gimenez:tesis} Gimenez JM (2015) Enlarging time-steps for solving one and two
phase flows using the particle finite element method. Ph.D. Thesis,
Universidad Nacional del Litoral, Santa Fe, Argentina
\bibitem{codina-soto} Ramon Codina, Orlando Soto,
Approximation of the incompressible Navier–Stokes equations using orthogonal subscale stabilization and pressure segregation on anisotropic finite element meshes,
Computer Methods in Applied Mechanics and Engineering,
Volume 193, Issues 15–16,
2004,
Pages 1403-1419,
ISSN 0045-7825,
https://doi.org/10.1016/j.cma.2003.12.030.
\bibitem{codina-oss-press} Ramon Codina,
Stabilization of incompressibility and convection through orthogonal sub-scales in finite element methods,
Computer Methods in Applied Mechanics and Engineering,
Volume 190, Issues 13–14,
2000,
Pages 1579-1599,
ISSN 0045-7825,
https://doi.org/10.1016/S0045-7825(00)00254-1.
\bibitem{chorin}
A.J. Chorin, Math. Comput. 22,745 (1968)
\bibitem{temam}
R. Temam, Arch. Rat. Mech. Anal. 32, 377 (1969)
\bibitem{tesis-gimenez}  J. Gimenez, Enlarging time steps for solving one and two phase flows using the particle finite element
 method, Ph.D. thesis, Facultad de Ingenier\'{\i}a y Ciencias H\'{\i}dricas - Centro de Investigaciones en Mecanica
 Computacional. Santa Fe, Argentina (2015).
