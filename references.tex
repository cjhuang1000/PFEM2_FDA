\bibitem{cpi1} FDA’s ``Critical Path'' Computational Fluid Dynamics (CFD)/Blood Damage Project: Computational Round Robin problems
https://nciphub.org/wiki/FDA\_CFD
\bibitem{cpi} FDA Critical Path Initiative (CPI) \\https://www.fda.gov/scienceresearch/specialtopics/\\criticalpathinitiative/default.htm
\bibitem{fda_res} Hariharan P, Giarra M, Reddy V, Day SW, Manning KB, Deutsch S, Stewart SF, Myers MR, Berman MR, Burgreen GW, Paterson EG (2011). Multilaboratory particle image velocimetry analysis of the FDA benchmark nozzle model to support validation of computational fluid dynamics simulations. Journal of Biomechanical Engineering, 133(4):041002-1.
\bibitem{fda_nozzle} Herbertson LH, Olia SE, Daly A, Noatch CP, Smith WA, Kameneva MV, Malinauskas RA (2015). Multilaboratory Study of Flow-Induced Hemolysis Using the FDA Benchmark Nozzle Model. Artificial Organs, 39(3):237--248.
\bibitem{fda_pump} Giarra MN (2009). Shear Stress Distribution and Hemolysis Measurements in a Centrifugal Blood Pump. Rochester Institute of Technology. Master Thesis, Rochester Institute of Technology. Rochester, New York.
\bibitem{fda_numrob} Zmijanovic V, Mendez S, Moureau V, Nicoud F (2017). About the numerical robustness of biomedical benchmark cases: Interlaboratory FDA's idealized medical device. International Journal for Numerical Methods in Biomedical Engineering, 33(1):e02789.
\bibitem{hariharan_nozzle} Hariharan P, D'Souza GA, Horner M, Morrison TM, Malinauskas RA, Myers MR (2017). Use of the FDA nozzle model to illustrate validation techniques in computational fluid dynamics (CFD) simulations. PloS One, 12(6):e0178749.
\bibitem{nassau_pump} Nassau CJ, Wray TJ, Agarwal RK (2015). Computational Fluid Dynamic Analysis of a Blood Pump: An FDA Critical Path Initiative. In ASME/JSME/KSME 2015 Joint Fluids Engineering Conference (pp. V002T26A002-V002T26A002). American Society of Mechanical Engineers.
\bibitem{heck_hemo} Heck ML, Yen A, Snyder TA, O'Rear EA, Papavassiliou DV (2017). Flow-Field Simulations and Hemolysis Estimates for the Food and Drug Administration Critical Path Initiative Centrifugal Blood Pump. Artificial Organs, 41(10):E129--E140. doi: 10.1111/aor.12837.
\bibitem{stewart_cfd} Stewart SF, Paterson EG, Burgreen GW, Hariharan P, Giarra M, Reddy V, Stewart SF, Paterson EG, Burgreen GW, Hariharan P, Giarra M, Reddy V, Day SW, Manning KB, Deutsch S, Bermand, MRm, Myers MR (2012). Assessment of CFD performance in simulations of an idealized medical device: results of FDA's first computational interlaboratory study. Cardiovascular Engineering and Technology, 3(2):139--160.
\bibitem{mali_cfd} Malinauskas RA, Hariharan P, Day SW, Herbertson LH, Buesen M, Steinseifer U, Aycock KI, Good BC, Deutsch S, Manning KB, Craven BA (2017). FDA benchmark medical device flow models for CFD validation. ASAIO Journal, 63(2):150--160.
\bibitem{gallagher_exp} Gallagher MB, Aycock KI, Craven BA, Manning KB (2018). Steady flow in a patient-averaged inferior vena cava—part I: particle image velocimetry measurements at rest and exercise conditions. Cardiovascular Engineering and Technology, 9(4):641--653.
% Gallagher, M.B., Aycock, K.I., Craven, B.A. et al. Cardiovasc Eng Tech (2018) 9: 641. https://doi.org/10.1007/s13239-018-00390-2
\bibitem{craven_cfd} Craven BA, Aycock KI, Manning KB (2018). Steady Flow in a Patient-Averaged Inferior Vena Cava-Part II: Computational Fluid Dynamics Verification and Validation. Cardiovascular Engineering and Technology, 9(4):654--673.
%Craven, B.A., Aycock, K.I. \& Manning, K.B. Cardiovasc Eng Tech (2018) 9: 654. https://doi.org/10.1007/s13239-018-00392-0
\bibitem{sph} Monaghan JJ (1988). An introduction to SPH. Computer Physics Communications, 48:89--96.
\bibitem{pic} Harlow FH (1955). A machine calculation method for hydrodynamic problems. Los Alamos Scientific Laboratory Report LAMS-1956.
\bibitem{mac} Harlow FH, Welch J (1965). Numerical calculation of time dependent viscous incompressible flow of fluid with free surface. Physics of Fluids, 8(12):2182--2189.
\bibitem{mpm} Wieckowsky Z (2004). The material point method in large strain engineering problems. Computer Methods in Applied Mechanics and Engineering, 193(39):4417--4438.
\bibitem{mps} Koshizuka S, Oka Y (1996). Moving particle semi-implicit method for fragmentation of incompressible fluid. Nuclear Science and Engineering, 123:421--434.
\bibitem{sergio:pfem} Idelsohn SR, O\~nate E, Del Pin F (2004). The particle finite element method a powerful tool to solve incompressible flows with free surfaces and breaking waves. International Journal for Numerical Methods in Engineering, 61:964--989.
\bibitem{sergio:xivs1} Idelsohn SR, Nigro NM, Limache A, O\~nate E (2012). Large time-step explicit integration method for solving problems with dominant convection. Computer Methods in Applied Mechanics and Engineering 217-220:168--185.
\bibitem{sergio:xivs2} Idelsohn SR, Nigro NM, Gimenez JM, Rossi R, Marti J (2013). A fast and accurate method to solve the incompressible Navier--Stokes equations. Engineering Computations, 30(2):197--222.
\bibitem{gimenez:parallel} Gimenez JM, Nigro NM, Idelsohn SR (2014). Evaluating the performance of the particle finite element method in parallel architectures. Computational Particle Mechanics, 1(1):103--116.
\bibitem{conjgrad} Hestenes MR, Stiefel E (1952). Methods of Conjugate Gradients for Solving Linear Systems. Journal of Research of the National Bureau of Standards. 49(6):409--436. doi:10.6028/jres.049.044.
\bibitem{sergio:pfem2_lts} Idelsohn SR, Marti J, Becker P, O\~nate E (2014). Analysis of multifluid flows with large time steps using the particle finite element method. International Journal for Numerical Methods in Fluids, 75(9):621--644.
\bibitem{gimenez:fs} Gimenez JM, Gonzlez LM (2015). An extended validation of the last generation of particle finite element method for free surface flows. Journal of Computational Physics, 284:186--205.
\bibitem{pablo:FSI} Becker P, Idelsohn SR, O\~nate E (2014). A unified monolithic approach for multi-fluid flows and fluid-structure interaction using the particle finite element method with fixed mesh. Computational Mechanics, 55(6):1091--1104.
\bibitem{lsdyna} LS-DYNA Manual.\\
http://www.lstc.com/download/manuals .
\bibitem{gimenez:st} Gimenez JM, Nigro N, O\~nate E, Idelsohn S (2016). Surface tension problems solved with the particle finite element method using large time-steps. Computers and Fluids, 141:90--104.
\bibitem{gimenez:tesis} Gimenez JM (2015). Enlarging time-steps for solving one and two phase flows using the particle finite element method. Ph.D. Thesis, Universidad Nacional del Litoral, Santa Fe, Argentina.
\bibitem{codina-soto} Codina R, Soto O (2004). Approximation of the incompressible Navier–Stokes equations using orthogonal subscale stabilization and pressure segregation on anisotropic finite element meshes. Computer Methods in Applied Mechanics and Engineering, 193(15-16):1403--1419, https://doi.org/10.1016/j.cma.2003.12.030.
\bibitem{codina-oss-press} Codina R (2000). Stabilization of incompressibility and convection through orthogonal sub-scales in finite element methods. Computer Methods in Applied Mechanics and Engineering, 190(13-14):1579--1599, https://doi.org/10.1016/S0045-7825(00)00254-1.
\bibitem{chorin} Chorin AJ (1968). Numerical Solution of the Navier-Stokes Equations. Mathematics of Computation, 22(104):745--762.
\bibitem{temam} Temam R (1969). On the Approximation of the Solution of Navier-Stokes Equations by the Fractional Steps Method II. Archive for Rational Mechanics and Analysis, 32:377--385.
%\bibitem{tesis-gimenez}  Gimenez J (2015). Enlarging time steps for solving one and two phase flows using the particle finite element method, Ph.D. thesis, Facultad de Ingenier\'{\i}a y Ciencias H\'{\i}dricas - Centro de Investigaciones en Mecanica Computacional. Santa Fe, Argentina (2015).
\bibitem{metis1} Karypis G, Kumar V. METIS, a Software Package for Partitioning Unstructured Graphs and Computing Fill-Reduced Orderings of Sparse Matrices.
\bibitem{metis} Karypis G, Kumar V (1998). A fast and high quality multilevel scheme for partitioning irregular graphs. SIAM Journal on Scientific Computing, 20(1):359--392.
\bibitem{gimenez-difusion} Gimenez JM, Nigro NM, Idelsohn SR (2012). Improvements to solve diffusion-dominant problems with PFEM-2. Mecanica Computacional, vol. XXXI:137--155.
\bibitem{Smirnov2001} Smirnov A, Shi S, Celik I (2001). Random flow generation technique for large eddy simulations and particle-dynamics modeling. Journal of Fluids Engineering, 123(2):359--371.
\bibitem{wale} Nicoud F, Ducros F (1999). Subgrid-Scale Stress Modelling Based on the Square of the Velocity Gradient Tensor Flow. Turbulence and Combustion, 62(3):183--200.
