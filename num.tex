The numerical scheme used in the approximation of the system of Eq. (\ref{eq:mom}-\ref{eq:cont}) is based on FEM. The Galerkin method will be used with a $p1-p1$ interpolation space for the pressure-velocity pair. Furthermore the discrete model will be decoupled by means of a fractional step method first propoused by Chorin \cite{chorin} and Temam \cite{temam}. To simplify the presentation a backward difference time integration scheme was used for the time derivative of $\vect{u}$ but another higher order scheme may also be used. 
%
\begin{align}\label{eq:masamat}
&\rho (\textbf{M}\vect{u}^\star-\textbf{M}\hat{\vect{u}}^n)/\Delta t+\mu\textbf{K}\vect{u}^{\star} +\beta\rho\textbf{S}(\bar{\vect{u}}^{n+1})\vect{u}^{\star}  = \nonumber\\&-\textbf{G}p^n+ \textbf{F},
\\\label{eq:matpres2}
& \textbf{L}p^{n+1}=\frac{\rho}{\Delta t}\textbf{D}\vect{u}^\star+ \textbf{L}  p^{n},
\\\label{eq:matunm2}
&\rho \bar{\textbf{M}}\vect{u}^{n+1}=\rho \bar{\textbf{M}}\vect{u}^\star-\Delta t\textbf{G}(p^{n+1}- p^n),
\end{align}
%
where the standard assembled FEM matrices are \textbf{M} for the mass matrix, \textbf{K} is the stiffness matrix, \textbf{L} is the Laplacian matrix, \textbf{D} is the divergence, \textbf{G} is the gradient matrix and $\bar{\textbf{M}}$ is the lumped mass matrix. The vector \textbf{F} is the body force term. The matrix \textbf{S} is the advection matrix which depends on the fluid velocity. The switch $\beta$ will define whether the advection is done in a Eulerain way or Lagrangian by using PFEM-2. The variable $\vect{u}^\star$ is an intermediate velocity that is predicted using the pressure from $t^n$. As such it is not yet divergence free and it has to be corrected in the last step. When $\beta=1$ the advection term is used to transport the velocity and then it takes the value  $\hat{\vect{u}}^n=\vect{u}^n$ which is the velocity from the previous time step $t^n$. When $\beta=0$ the particles transport the velocity which has to be projected back onto the mesh nodes. More about this projection will be discussed later.

In this work standard $p1-p1$ elements are implemented which do not satisfy the Babuzka-Brezzi condition and thus pressure stabilization needs to be added. Although the fractional step methos adds sufficient numerical compressibility to make stable when time steps are relatively large the second order version of the method loses this property. In this work the Orthogonal Subgrid-Scale (OSS) method by Codina \cite{codina-oss-press} was implemented to prevent this problem. In a similar fashion the Galerkin approximation of the advection terms need to be stabilized for the FEM approximation only case. Again the ideas of Codina as described in \cite{codina-soto} are implemented and the OSS method is used. 
