A key step in the development of medical devices is the testing phase. Regulatory agencies like the Food and Drug Administration (FDA) require extensive laboratory testing and a long, tedious and expensive clinical trial process before a device is approved for clinical use. In an effort to optimize this process the industry and the regulatory agencies are looking at numerical methods as an additional tool that could potentially reduce the time of product development, animal testing and cost. A first critical step to do this end is to increase the confidence, reliability and robustness of numerical techniques. The FDA is actively working on this subject by designing laboratory experiments that could be used to evaluate the solution accuracy provided by computational methods launching the Critical Path Initiative (CPI) \cite{cpi} program. The aim of the program is to standardize the use of computational simulation on the design of the blood-contacting medical
devices and analysis of the ratio of hemolysis in them. The goal of this project is to establish the guidelines for
applying CFD on the evaluation and the optimization of the medical devices. FDA has proposed two benchmark
problems \cite{cpi1} for CFD verification and validation. The first benchmark problem is the flow in a nozzle
containing a gradual and sudden change of the diameter. The flows at different flow rates which
correspond to different flow regimes are examined. The second study is the flow in a simplified centrifugal
blood pump. The flow field under various pump operation conditions are analyzed. For each benchmark
problem, the experimental results \cite{fda_res,fda_nozzle,fda_pump} and the flow field predicted with numerical simulations \cite{fda_numrob,hariharan_nozzle,nassau_pump,heck_hemo} from
different institutes are collected. The comparison between results obtained using different numerical models are
made and analyzed \cite{stewart_cfd,mali_cfd}. 

Additionally a third benchmark problem will be presented which involves the analysis of steady flow in a patient-averaged inferior vena cava \cite{gallagher_exp,craven_cfd}. Although there are many numerical results in this area only a few compare with flow measurements of experimental results. In particular the study represents an anatomical model of the  inferior vena cava (IVC) that includes the primary morphological features that influence the hemodynamics (iliac veins, infrarenal curvature, and non-circular vessel cross-section).

In this study, these three benchmark problems will be presented and contrasted with numerical results using the Finite Element Method (FEM) and the enhanced Particle Finite Element Method (PFEM-2).

The main goal of this paper is to evaluate the performance of the PFEM2 method in the field of biomedical devices comparing the predictions with the experimental results. The PFEM2 method is based on the idea that advection effects are approximated in a Lagrangian way using particles. Many numerical methods are based on these ideas \cite{sph,pic,mac,mps,mpm} including the early version of PFEM \cite{sergio:pfem}. In the latter the advection and all derivatives were computed on a finite element mesh that had to be re-constructed at every time step due to the Lagrangian nature of the method. All the fields were approximated using the FEM. The PFEM provides excellent results in problems with free surface or fluid structure interaction but it had poor performance in problems involving internal/external aerodynamics when compared to Eulerian methods due to the additional mesh operations which in turn also force a complete re-factorization of all linear systems. The work of Idelsohn et. al. \cite{sergio:xivs1,sergio:xivs2} introduced a new integration strategy called X-IVS that employed a fixed mesh modifying the PFEM and creating PFEM-2. Not only these improvements made PFEM-2 competitive in problems classically solved using Eulerian methods but it provided the advantage that larger time steps could be used \cite{gimenez:parallel} reducing the computational time in advection dominated problems and eliminating the traditional advection stabilization terms known to introduce additional numerical diffusion. By eliminating the advection term a fractional step method provides a fully decoupled momentum equation among the three velocity components which saves storage and simplify the implementation. The left hand side also becomes symmetric allowing the use of simpler linear algebra solvers like Conjugate Gradient \cite{conjgrad}. The method has also been successfully implemented for multi-phase flows \cite{sergio:pfem2_lts,gimenez:fs,gimenez:tesis}, problems involving surface tension \cite{gimenez:st} and fluid structure interaction \cite{pablo:FSI}.

There are many applications in the biomedical field that could benefit from the capabilities of PFEM-2. In particular those applications that involve long real time simulations like drug delivery where a fluid component is injected in another fluid. In this case it is also important to keep track of the sharp interface at the drug advancing front. {MORE APPLICATIONS}

In this work the implementation of PFEM-2 was performed in the commercial software LS-DYNA\textsuperscript{\textregistered} which is a multi-physics solver for non-linear dynamics. The module used for the implementation was ICFD which deals with incompressible fluid flows. 

The rest of this paper will be organized as follows. 
In the first section the equations governing the flow of incompressible fluids will be presented together with boundary and initial conditions. 
In section two the PFEM-2 method will be introduced and some implementation aspects will be discussed.
In section three the time and space discretization of the equations of motion will be presented. 
In section four the benchmark problems will be introduced and the experimental and numerical results will be presented.
