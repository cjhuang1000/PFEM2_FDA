\textbf{NEED REVISION}:*** In the present study, the numerical scheme of PFEM-2 was implemented and applied on biomedical problems. In PFEM-2, the advection term is dealt separately with the transportation of Lagrangian particles. These Lagrangian particles moves with local velocities following the X-IVS integration scheme. The rest of the governing equation was tackled with fractional step method on fixed finite element mesh. The advantage of PFEM-2 is that it allows the use of larger time step size, even when CFL number larger than $1$, without compromising the stability. The parallel implementation was introduced when particles moving across multi-processors during advection stage, and the scalability performance with different inventory strategies were presented. 

The PFEM-2 was applied on biomedical field of blood nozzle, blood pump, and IVC flows. The results of PFEM-2 along with the results using classical FEM were analyzed and compared with experiment measurement from literature. Firstly, the nozzle flow at Re $3500$ is simulated with different meshes and time step sizes. The resultant jet breakdown locations using FEM differ largely between different meshes and CFL numbers, while PFEM-2's results differ little and still have good agreement with experiments even with $CFL=10$. As to the flow in simplified blood bump geometry, numerical simulations were performed under the non-inertial frame approximation. The pressure difference estimated by FEM has $3.6$\% of discrepancies with experiment, but PFEM-2 predicted a much higher pressure difference across the pump. It was observed the streamline between blades by PFEM-2 don't turn accordingly following the shape of the chamber, and further investigation is needed for improvement. Lastly, the flow in patient-averaged IVC geometry was analyzed at resting and exercising conditions. The grid convergence study using FEM showed a reasonable convergence rate with the proposed meshes. The quantitative error of the in-plane velocity on selected sagittal and coronal planes were computed with different numerical approaches. The FEM and PFEM-2 obtained nearly the same error at resting condition, and PFEM-2 got better result on coronal plane at exercising condition.
***
