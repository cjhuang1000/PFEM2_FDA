In the present study, the numerical scheme of PFEM-2 was implemented and applied to biomedical problems. In PFEM-2, the advection term is dealt separately by transporting massless Lagrangian particles. These Lagrangian particles move with local velocities following the X-IVS integration scheme. The rest of the governing equations were tackled with a fractional step method on a fixed finite element mesh. The advantage of PFEM-2 is that it allows the use of larger time step size allowing CFL number much larger than $1$ without compromising the stability and accuracy. The parallel implementation was introduced to show how particles can move across multi-processors during the advection stage, and the scalability performance with different inventory strategies was presented. 

The PFEM-2 was applied to benchmark problem in the biomedical field for a blood nozzle, blood pump, and IVC flows. The results of PFEM-2 along with the results using a classical FEM formulation were analyzed and compared with experimental measurements from literature. Firstly, the nozzle flow at Re $3500$ is simulated with different meshes and time step sizes. The resultant jet breakdown locations using FEM differ largely between different meshes and CFL numbers, while PFEM-2's results differ little preserving a good agreement with experiments even for $CFL=10$. Secondly the flow in a simplified blood bump geometry was studied, numerical simulations were performed under the non-inertial frame approximation. The pressure difference estimated by FEM has a $3.6$\% of discrepancies with the experimental value while PFEM-2 predicted a lower pressure difference across the pump with an error of $14.6$\%. Although the pressure differential error is higher for PFEM-2 the velocity profile at different cross sections of the pump are in better agreement with the experimental observations than the velocity values obtained with the FEM. Lastly, the flow in a patient-averaged IVC geometry was analyzed at resting and exercising conditions. The grid convergence study using the FEM formulation showed a reasonable convergence rate with the proposed meshes. The quantitative error of the in-plane velocity on selected sagittal and coronal planes were computed with different numerical approaches. Both FEM and PFEM-2 obtained nearly the same error at resting condition while PFEM-2 obtained a better result on a coronal plane at exercising condition.

As a final conclusion it is evident from the results that PFEM-2 shows less sensitivity to both mesh configuration and time step size when compared to the FEM formulation presented in this paper. The advantage of PFEM-2 becomes more evident in problems of higher Reynolds number as it was shown in the analysis of the IVC flow.

In a future development a more efficient inventory strategy will be implemented that will improve the scalability of PFEM-2 to make it more competitive when compared to that of the FEM formulation.
