In recent years the biomedical industry has shown increased interest in using numerical methods to assist in the R\&D of medical devices. The long term goal is to reduce the costly and lengthy process that clinical trials take to for the Food and Drug Administration (FDA) to approve a device. For this both FDA and academia are working together to creat laboratory experiment that will help the industry gain confidence in numerical techniques as well as provide software developers with insights on the deficiencies that numerical softeware may have. In this article three bechmarks proposed by the FDA are used to compare experimental results with those of the Finite element method (FEM) and Particle Finite Element Method 2 (PFEM-2). The first benchmark problem is the flow in a nozzle
containing a gradual and sudden change of the diameter. Several flow regimes are studied. The second problem studies the flow in a simplified centrifugal
blood pump under various pump operation conditions. Finally the third benchamrk studies the steady flow in a patient-averaged inferior vena cava. PFEM-2 is regarded as tool with great potential mainly because no stabilization is needed for the Galerkin approximation of the advection term in the transport equations. This could be a big advantage in problems with large gradients as it is the case for flows at high Reynolds number. This paper is an effort to test PFEM-2 in real world engineering applications.
